% latex comview.tex ; dvips comview.dvi -o comview.ps 
% ghostview foo.ps

\documentstyle[12pt,epsf]{article}
\input{epsf.tex}

\newcommand{\comview}{{\tt comview}}

% For TCA
% \newcommand{\system}{TCA}
% For IPC
\newcommand{\system}{IPC}

\begin{document}

\title{Comview User's Guide}
\author{Version 1.0}
\maketitle

\tableofcontents

\section{Introduction}

\comview{} is a tool designed for visualizing the message-passing communication
that occurs between modules in \system{}-based systems. It can serve as a
useful tool for debugging and monitoring a \system{} session (while the
session is active, or for later analysis).

\comview{} uses a log file generated from a session with \system{}'s central
server. For each message, central logs the source and destination
modules, the time, the name, and type of the message. This log file is
parsed and processed by \comview{} to generate a graphical display
showing the communication that occurs between modules over time. The contents of
the log file are also shown within a section of the display.

The graphical display consists of widgets that can be selected to see
more information about the chosen activity. There is a link between
each widget in the graphical display and the message it corresponds to
in the log file. Selecting an object in the graphical display will
highlight the corresponding line within the log file, and vice versa.

\comview{} is designed to make it easy to search for particular types of
message activity, through a number of different mechanisms. A list of all
the names of messages sent is maintained, in addition to a list of all
modules that have sent or received messages. These lists can be viewed
and individual messages from them can be selected. Additionally, the log file
itself can be searched using a simple text-search. Configuration files
can also be used to suppress the display of particular messages or
modules.

\section{Starting Comview}

Since \comview{} uses Tcl/Tk, the location of these libraries must be
specified in the user's environment variables (TCL\_LIBRARY and
TK\_LIBRARY). Currently, \comview{} uses Tcl version 7.4 and Tk version
4.0.

\subsection{Logging with central}

To create a log file, the \system{} central server must be run with at least
the following logging options: ``{\tt-Lmtirn -f<logfile>}'' (m=message
traffic, t=time messages are handled, i=ignore logging registration
and deregistration messages, r=log the reference id as well as the
message name, and n=do not prompt the user for comments). You may use any other options except ``{\tt-Lx}'' (no
logging). Type ``{\tt central -h}'' for more information on central's
command line options.

In order to be able to view the data associated with messages while
using \comview{} you must additionally give the ``{\tt d}'' logging option to
the central server (e.g. ``{\tt -Ldmtirn}'').

\subsection{Command line options}

\comview{} is run as follows: {\tt comview [-options] -f <logfile>}

{\tt
lung\% comview -h\newline
-h : Print this message\newline
-v : Print version information\newline
-l : Do not display all log file information (only messages)\newline
-f <logfile>: Use the logfile\newline
-z <zoom>: initial zoom value}\newline

The command line options ``{\tt -h}'' and ``{\tt -v}'' are self-explanatory.
The ``{\tt -l}'' option refers to the part of the \comview{} display (the
textual log-window) that
displays lines directly read in from the \system{} log
file. The ``{\tt -l}'' option indicates that the log-window should contain
only lines that correspond to valid messages in the
graphic display. Without this option, the log-window will contain all the
information that \system{} logs, such as
the data associated with individual messages.

The ``{\tt -z}'' option is for specifying an initial zoom value (a
floating point number). The default setting (specified in
options.tcl) is 10.0, which results in an initial canvas width of 8 seconds.

\section{Using Comview}

\subsection{Stepping and Running}

\begin{figure}[htp]
\centerline{\epsfysize=6in \epsfbox{comview1.ps}}
\caption{Comview Display}
\label{fig:comview}
\end{figure}

\comview{} displays information in two
places: in a textual log window at the bottom of the display and in the larger
graphical trace window called the canvas display (named after the Tk
object which is used to create it). Figure~\ref{fig:comview} shows a
snapshot of the entire \comview{} screen.

To visualize message traffic, one can either ``Step'' through
or ``Run'' through the log file. Stepping will read in a single line
from the log file, parse it, and display activity in the log window and canvas
display.  Running through will continuously step through the log file
until ``Run'' is hit again, or until the end of the log file is reached.

\comview{} can use a log file that is being produced while it is
running. If the end of the log file is reached then the ``Run'' button
will change from green (running) to yellow (waiting). If more data is
written to the log file, \comview{} will resume running.

The canvas display expands and scrolls horizontally to follow incoming data.
The automatic scrolling feature is toggled by clicking on the ``Scroll-Lock''
button.  With ``Scroll-Lock'' on, the canvas display does not scroll, but the
graphics are still updated as messages are read in.  The initial setting is for
the display to scroll so that newly created activity is visible.

The log window consists of the individual lines from the log
file. Unless the ``{\tt -l}'' option is given to \comview{}, this
consists of all lines from the log file, such as those that
contain the data associated with a message, and lines which
report on the internal performance of \system{} execution. If the ``{\tt
-l}'' option is given to \comview{}, then the log window will display
only lines from the log file that indicate message activity.

When the log file indicates that a new module has connected with central,
a label for that module 
will appear on the left side of the screen. As the log file reports that a
module sends or receives messages, the color of the module label changes
to indicate 
what kind of message the module is currently processing or what kind
of message that module has just sent.

As different messages are sent from or received by a module, a colored
rectangle is added to the right side of the module's display line. The
width of the rectangle is proportional to the time it takes for the message to
be handled.

The various colors in the display have the following meanings:

\begin{center}
\begin{tabular}{|l|l|r|} \hline
{\em message type}	& {\em sender's color}	
				& {\em recipient's color} \\ \hline
Query		& Yellow		& Green \\
Inform		& Green			& Dark Green \\
Command		& Light Blue		& Blue \\
Broadcast	& Light Purple		& Purple \\
Goal		& Light Olive		& Olive \\ \hline
\end{tabular}
\end{center}

These colors are specified (and can be changed) from the file
``options.tcl''. The colorings allow you to look at a segment of activity
for a particular module and be able to identify the state of the
module at that time (e.g., waiting for a message, or processing a
particular message type).

\subsection{Selecting Objects}

Any of the objects within the canvas can be selected by clicking in
the canvas window with the left mouse button. This will result in
arrows being drawn which show the communication pathways of the chosen
activity. One can also double-click on a line in the log window and
the appropriate activity (if any) within the canvas window will be
highlighted.

Blocking and non-blocking messages will result in different types
of highlighting.
For blocking messages (such as query messages) the module that sent
the message is going to wait for a reply. In the log file there will
be a reply message. Graphically, highlighting will draw an arrow from
the source module to the destination module at the time at which the
query was sent. At the time at which the reply was sent an arrow will
be drawn from the destination module to the source module. This is
shown in figure~\ref{fig:query}.

\begin{figure}[htp]
\centerline{\epsfysize=2in \epsfbox{query.ps}}
\caption{Highlight of Query Message}
\label{fig:query}
\end{figure}

In contrast, non-blocking messages (such as broadcast or inform messages) are
sent and do not
wait for a reply. Graphically, this is represented as a short segment
of activity (of width 1/100th of a second) for the source module. The
destination module becomes active and will remain active until it
completes handling the given message (this is identified within the
log file by a ``Done'' message). Figure~\ref{fig:broadcast} shows
highlighting for an broadcast message.

\begin{figure}[htp]
\centerline{\epsfysize=2in \epsfbox{inform.ps}}
\caption{Highlight of Broadcast Message}
\label{fig:broadcast}
\end{figure}

If the central server receives a message for a module that is
currently processing another message, it queues these messages until
the module has completed handling its current activity before sending
the next message. Graphically, this is represented by creating an
orange ``pending bar'' above the other activity for the module (again,
the color of the pending bar is reconfigurable). The pending bar will
remain until all queued messages have been handled. Unfortunately,
\comview{} currently has no way to identify when more than one message is
pending for a given module.

\begin{figure}[htp]
\centerline{\epsfysize=2.0in \epsfbox{pend.ps}}
\caption{Highlight of a Message That Pends}
\label{fig:pend}
\end{figure}

Figure~\ref{fig:pend} shows the calling structure for a query message
that pends. The first vertical arrow indicates when the message is
sent, the horizontal arrow indicates when the message is forwarded to
the module to be handled, and the downward pointing arrow corresponds
to the reply message. A similar graphical display is created for
non-blocking messages (but, without the reply arrow).

\subsection{Rearranging Modules}

The top-to-bottom order of the modules in the graphic display can be changed
(by default, they appear in the order that they connect to central). This
is done by pressing the left mouse button on a module name and then
dragging it to the new position. This will shift the order of the
other modules to fit the moved module to the new location.

\subsection{Getting Around the Canvas}

\comview{} has several mechanisms to obtain different views of the
canvas. There is a scroll-bar at the bottom of the canvas for manually
scrolling through the canvas. Clicking on a line in the log-window selects the
corresponding object, and jumps the canvas window to make the selected object
visible. 
Zooming provides, as expected, a means of scaling the objects on the canvas.

\subsubsection{Zoom}

The ``Zoom In'' and ``Zoom Out'' buttons (Figure~\ref{fig:comview}) will zoom,
centered around the currently
selected object. If no object is selected, then the zoom will be about
the center of the current view of the canvas. If an object is selected
and you want to zoom about the center of the canvas, use the
``Unselect'' button before any zoom.

The zoom slider widget is another way of zooming about the
canvas. When set to 1.0 the entire canvas is visible (maximum zoom
out). Set to 0.0 the canvas is zoomed in to a fixed scale where 1
pixel corresponds to 1/100th of a second (it is possible to zoom in
more by using the ``Zoom In'' button).

\subsection{Time Displays}

There are three displays of time on the screen. The clock below the
``Unselect'' button (Figure~\ref{fig:comview}) shows the time that
corresponds to the mouse position when it is over the canvas. Above
the canvas on the top-left and top-right are two more clocks that
indicate the time that corresponds to the leftmost visible part of the
canvas and the rightmost visible part of the canvas.

These three clocks can be displayed in absolute units (the time-stamp
that central logs with every message) or relative time units (such
that the time at the leftmost point in the canvas will be
00:00:00.00). The ``Time'' menu has commands for toggling between the
units.

\section{Menu Items}

At the top of the \comview{} display is a menu bar with five items: ``File'',
``Message'', ``Module'', ``Time'' and ``Jump''.  This section describes each
of the menu choices, in turn.

\subsection{File Menu}

\subsubsection{Saving canvas to postscript}

The canvas can be saved to postscript using the ``Print Current View''
option under the ``File'' menu. This will open a window prompting for
an output filename. The view saved is the currently visible portion of
the canvas window. The view will be saved in landscape orientation.

\subsection{Message Menu}

Once a log file has been loaded, users will typically need to search through
the log file to find particular messages or events. \comview{} provides a number
of different features that make this possible.

The ``Message'' menu consists of the names of all the different types of
messages that have 
been sent.  This menu item is updated as more of the log file is read in.
Selecting a particular message from this menu will bring up
a new window that lists all the occurrences of that 
message type, as well as the number of occurrences of that message (this list
is updated as more of the log file is processed). Any particular instance can
be selected and the corresponding 
activity within the graphic display will be highlighted.  

For example, Figure~\ref{fig:msglist} shows the sixty-four ATME\_RST messages
that were 
sent (in a scrollable window). The highlighted message is line 2 of the log
file, and indicates that the message was broadcast from module acs\_r1 to the
console module at time 14:34:56.21.

\begin{figure}[htp]
\centerline{\epsfysize=2in \epsfbox{msglist.ps}}
\caption{Message List Window}
\label{fig:msglist}
\end{figure}

The ``Show All'' button on the bottom of the message list window highlights all
the occurrences of that message name within the
canvas. Figure~\ref{fig:showall} shows a view of the canvas after ``Show All''
has been selected for the ATBE\_RST messages shown in
figure~\ref{fig:msglist}.

\begin{figure}[htp]
\centerline{\epsfysize=3.5in \epsfbox{showall.ps}}
\caption{Canvas View After ``Show All''}
\label{fig:showall}
\end{figure}

\subsection{Module Menu}

In a similar manner, the ``Module'' menu lists all the different modules that
have connected to the central server. Selecting a particular module from this
menu opens a new window that lists all the messages that have been sent to
or received by the chosen module, as well as the number of message instances that
involve the chosen module.  Again, this list is updated as more of the log
file is processed.  As with the ``Message'' menu, these messages can be
individually selected, or all of them highlighted at once using the ``Show
All'' button. Figure~\ref{fig:showall} shows the window that is created
when the ``console'' module is selected from the ``Module'' menu.

\begin{figure}[htp]
\centerline{\epsfysize=2in \epsfbox{modlist.ps}}
\caption{Module List Window}
\label{fig:modlist}
\end{figure}

\subsection{Time Menu}

The first two options in the time menu, ``Absolute'' and ``Relative''
toggle the units that are used for the three time displays. ``Absolute''
selects the units that corresponds to the time-stamp that central logs
with every message. Relative units are a measure of time since the
first message was logged (such that the time at the leftmost point in
the canvas will be 00:00:00.00).

The ``Display Visible Time'' menu option toggles whether the
time at the top-left and top-right positions above the canvas are
visible (this may be useful on slower machines).

\subsection{Jump Menu}

The ``Jump'' menu is used to easily access particular messages in the
log and canvas windows. The ``Jump'' menu has two features: A
selection to jump to a given time, and a selection to jump to a
specific message number.

Selecting ``Time'' from this menu will bring up a new window that shows the
time of 
the first logged message and the time of the last logged message (see
Figure~\ref{fig:gototime}).

\begin{figure}[htp]
\centerline{\epsfysize=1.3in \epsfbox{gotime.ps}}
\caption{Go-To Time Window}
\label{fig:gototime}
\end{figure}

Any time between those two times (inclusive) can be entered. When
``Go'' is pressed the canvas will be shifted so that it is centered
around the chosen time.

The ``Message Number'' option opens a new window that shows the
maximum message number and has an entry in which to type a message
number to jump to (see Figure~\ref{fig:gotomsg}). The chosen message
will be highlighted and centered in both the canvas window and the log
window.

\begin{figure}[htp]
\centerline{\epsfysize=0.7in \epsfbox{gomsg.ps}}
\caption{Go To Message Window}
\label{fig:gotomsg}
\end{figure}

Both of these jump commands can be performed using keyboard-only
commands (next section).

\section{Keyboard Shortcuts}

There are a number of keyboard shortcuts that allow you to step
through the log file and execute several of the same commands that you
can select via the menus. The following table describes them:

Keyboard Control
\begin{tabular}{|l|r|} \hline
{\em function}			& {\em keypress} \\ \hline
highlight next message 		& control-n \\
highlight previous message  	& control-p \\
highlight first message		& alt-p \\
highlight last message 		& alt-n \\
scroll forward one screen on canvas & spacebar \\
scroll backward one screen on canvas & delete \\
run 				& alt-r \\
step	 			& alt-s \\
zoom in 	 		& ctrl-i \\
zoom out 			& ctrl-o \\
goto message number 		& alt-g \\
goto time	 		& alt-t \\
search log window	  	& ctrl-s \\ 
reverse search log window  	& ctrl-r \\ 
mini-buffer quit	 	& ctrl-g \\ \hline
\end{tabular}

\section{Searching the Log Window}

\comview{} includes a feature for textually searching through the processed
log file.  This feature is accessible only via the keyboard (there
is no menu option for it). One can search the log file from the
current position to the end. This is invoked by pressing Control-S,
and it operates in a manner similar to an emacs mini-buffer (see
Figure~\ref{fig:search}).

\begin{figure}[htp]
\centerline{\epsfysize=2in \epsfbox{search.ps}}
\caption{Searching the Log Window}
\label{fig:search}
\end{figure}

If the search fails, the mini-buffer will blink. If the search is
successful, the highlight-bar in the log file will move to the line
that contains the text which matches the search string. Pressing
Control-S again searches from the new position to the end of the log
file. Control-G terminates the search.

In a similar manner, Control-R will search the log window in the reverse
direction.

Alt-P moves the current log window selection line to the first message in
the log window. This highlights the first message and
shifts the canvas so that the appropriate graphic is visible in the
canvas window.  Any search will now begin from the top of the log file. To
jump to the end of the log file, press Alt-N.

\section{Settings File}

Certain features of \comview{} can be specified in a settings file.  This file
must have the same filename as the log file, except with the extension
``.set'' instead of ``.log'' (i.e. ``test.log'' would have a settings file
``test.set'').  At present, the settings file can indicate to ignore
displaying of certain messages and/or certain modules.

\subsection{Filtering Out Specific Messages}

\comview{} has support for filtering out specific message names for a
given log file. The messages to filter out are listed in separate lines in the
settings file.  The settings file should have lines of the form:\newline
\newline
 ignore\_message message\_name1\newline
 ignore\_message message\_name2
\newline
\newline
\comview{} will then ignore any messages with the name message\_name1,
message\_name2, etc. These messages will neither appear in the log
window nor will they result in the creation of any graphical activity
in the canvas. Note that the settings file must exist at the time
\comview{} is started, and any changes after \comview{} has been started
will not be recognized.

\subsection{Ignoring Modules}

In a manner similar to the way messages can be ignored, messages that
involve a specific module can also be filtered out. All messages that
are sent to or from the selected module will be completely ignored
(they will not result in any graphical activity, and will not appear
in the log window).

The settings file should contain lines of the form:\newline
\newline
 ignore\_module module\_name1\newline
 ignore\_module module\_name2
\newline

\end{document}
